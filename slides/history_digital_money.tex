% Choose one to switch between slides and handout
\documentclass[]{beamer}
%\documentclass[handout]{beamer}

% Video Meta Data
\title{Bitcoin, Blockchain and Cryptoassets}
\subtitle{History of Digital Money}
\author{Prof. Dr. Fabian Schär}
\institute{University of Basel}

% Config File
% Packages
\usepackage[utf8]{inputenc} 
\usepackage{hyperref}
\usepackage{gitinfo2}
\usepackage{tikz}
\usepackage{amsmath}
\usepackage{bibentry}
\usepackage{xcolor}
\usepackage{caption}

% Beamer Template Options
\beamertemplatenavigationsymbolsempty
\setbeamertemplate{footline}[frame number]
\setbeamercolor{structure}{fg=black}
\setbeamercolor{footline}{fg=black}
\setbeamercolor{title}{fg=black}
\setbeamercolor{frametitle}{fg=black}
\setbeamercolor{item}{fg=black}
\setbeamercolor{}{fg=black}
\setbeamercolor{bibliography item}{fg=black}
\setbeamercolor*{bibliography entry title}{fg=black}
\setbeamertemplate{items}[square]
\setbeamertemplate{enumerate items}[default]
\captionsetup[figure]{labelfont={color=black},font={color=black}}
\captionsetup[table]{labelfont={color=black},font={color=black}}

\setbeamertemplate{bibliography item}{\insertbiblabel}

% Link Icon Command
\newcommand{\link}{%
    \tikz[x=1.2ex, y=1.2ex, baseline=-0.05ex]{% 
        \begin{scope}[x=1ex, y=1ex]
            \clip (-0.1,-0.1) 
                --++ (-0, 1.2) 
                --++ (0.6, 0) 
                --++ (0, -0.6) 
                --++ (0.6, 0) 
                --++ (0, -1);
            \path[draw, 
                line width = 0.5, 
                rounded corners=0.5] 
                (0,0) rectangle (1,1);
        \end{scope}
        \path[draw, line width = 0.5] (0.5, 0.5) 
            -- (1, 1);
        \path[draw, line width = 0.5] (0.6, 1) 
            -- (1, 1) -- (1, 0.6);
        }
    }

% Custom Titlepage
\defbeamertemplate*{title page}{customized}[1][]
{
  \vspace{-0cm}\hfill\includegraphics[width=2.5cm]{../config/logo_cif} 
  \includegraphics[width=1.9cm]{../config/seal_wwz} 
  \\ \vspace{2em}
  \usebeamerfont{title}\textbf{\inserttitle}\par
  \usebeamerfont{title}\usebeamercolor[fg]{title}\insertsubtitle\par  \vspace{1.5em}
  \small\usebeamerfont{author}\insertauthor\par
  \usebeamerfont{author}\insertinstitute\par \vspace{2em}
  \usebeamercolor[fg]{titlegraphic}\inserttitlegraphic
    \tiny \noindent \texttt{Commit Hash: \gitHash}\\ 
	\texttt{Commit Time: \gitAuthorIsoDate}\\ \vspace{1em}
  \link \href{https://github.com/cifunibas/Bitcoin-Blockchain-Cryptoassets/blob/main/slides/intro.pdf}
  {Get most recent version}\\
  \link \href{https://github.com/cifunibas/Bitcoin-Blockchain-Cryptoassets/blob/main/slides/intro.pdf}
  {Watch video lecture}\\ \vspace{1em}
  License: \texttt{Creative Commons Attribution-NonCommercial-ShareAlike 4.0 International}\\\vspace{2em}
  \includegraphics[width = 1.2cm]{../config/license}
}


%%%%%%%%%%%%%%%%%%%%%%%%%%%%%%%%%%%%%%%%%%%%%%
%%%%%%%%%%%%%%%%%%%%%%%%%%%%%%%%%%%%%%%%%%%%%%
\begin{document}

\thispagestyle{empty}
\begin{frame}[noframenumbering]
	\titlepage
\end{frame}

%%%
\begin{frame}{Before Bitcoin}

\begin{figure}
	\begin{tikzpicture}
	
			\draw 	
	
	\end{tikzpicture}
\end{figure}

This is how lists can be used in the template. \\ \vspace{1em}

Some of the main features of this template:
	
	\begin{itemize}
		\item<1-> stay focused
		\item<2-> save ink
		\item<3-> do not get distracted
	\end{itemize}

	\vspace{1em}	
	\uncover<4->{The same works for Enumerate environments:}

	\begin{enumerate}
		\item<5-> This is the first point
		\item<6-> and the second one
		\item<7-> Last but not least, the third one
	\end{enumerate}
	
\end{frame}
%%%	

%%%
\begin{frame}{Tables}
	\begin{table}
		\begin{tabular}{ll}
			A & B\\
			C & D
		\end{tabular}
		\caption{This is a table}
		\label{tbl:simpletable}
	\end{table}
\end{frame}

%%%
\begin{frame}{Equations}

Some in-line math can be used like this $y=x^2+2x-5$. \\ \vspace{1em}

However, in many cases it makes sense to use numbered equations. The $\&$-sign can be used to set anchor points, e.g. around the ``$=$'' for multiple equations. This is shown below in Equation \eqref{eq:firststep} and Equation \eqref{eq:secondstep}.
	\begin{align}
		y + 2x &= x^2+2x-5 \label{eq:firststep}\\
		y &= x^2-5 \label{eq:secondstep}
	\end{align}
\end{frame}
%%%

%%%
\begin{frame}{Figures}
	\begin{figure}
		\center
		\includegraphics[width=0.8\textwidth]{../config/logo_cif}	
		\caption{Center for Innovative Finance Logo}
		\label{fig:logo}
	\end{figure}
\end{frame}
%%%


%%%
\begin{frame}{Labels and (Non-Bib) References}
	In some cases in can make sense to reference to a figure, table or equation on a different slide. Use \texttt{$\backslash$pageref\{$<$label$>$\}} if you want to highlight, that there is a related resource on slide \pageref{fig:logo}.\\ \vspace{1em}
	
	Do - under no circumstances - hardcode references.	
\end{frame}
%%%


\begin{frame}{Bib-References}
		Read the Bitcoin Whitepaper, \cite{nakamotoBitcoin2008}.
\end{frame}


\begin{frame}%[allowframebreaks]
\frametitle{References and Recommended Reading}
	\bibliographystyle{amsplain}
	\bibliography{../assets/bib/refs}
\end{frame}


\end{document}