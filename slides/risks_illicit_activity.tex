% Choose one to switch between slides and handout
%\documentclass[]{beamer}
\documentclass[handout]{beamer}

% Video Meta Data
\title{Bitcoin, Blockchain and Cryptoassets}
\subtitle{Risks \& Illicit Activity}
\author{Prof. Dr. Fabian Schär}
\institute{University of Basel}

% Config File
% Packages
\usepackage[utf8]{inputenc} 
\usepackage{hyperref}
\usepackage{gitinfo2}
\usepackage{tikz}
\usepackage{amsmath}
\usepackage{bibentry}
\usepackage{xcolor}
\usepackage{caption}

% Beamer Template Options
\beamertemplatenavigationsymbolsempty
\setbeamertemplate{footline}[frame number]
\setbeamercolor{structure}{fg=black}
\setbeamercolor{footline}{fg=black}
\setbeamercolor{title}{fg=black}
\setbeamercolor{frametitle}{fg=black}
\setbeamercolor{item}{fg=black}
\setbeamercolor{}{fg=black}
\setbeamercolor{bibliography item}{fg=black}
\setbeamercolor*{bibliography entry title}{fg=black}
\setbeamertemplate{items}[square]
\setbeamertemplate{enumerate items}[default]
\captionsetup[figure]{labelfont={color=black},font={color=black}}
\captionsetup[table]{labelfont={color=black},font={color=black}}

\setbeamertemplate{bibliography item}{\insertbiblabel}

% Link Icon Command
\newcommand{\link}{%
    \tikz[x=1.2ex, y=1.2ex, baseline=-0.05ex]{% 
        \begin{scope}[x=1ex, y=1ex]
            \clip (-0.1,-0.1) 
                --++ (-0, 1.2) 
                --++ (0.6, 0) 
                --++ (0, -0.6) 
                --++ (0.6, 0) 
                --++ (0, -1);
            \path[draw, 
                line width = 0.5, 
                rounded corners=0.5] 
                (0,0) rectangle (1,1);
        \end{scope}
        \path[draw, line width = 0.5] (0.5, 0.5) 
            -- (1, 1);
        \path[draw, line width = 0.5] (0.6, 1) 
            -- (1, 1) -- (1, 0.6);
        }
    }

% Custom Titlepage
\defbeamertemplate*{title page}{customized}[1][]
{
  \vspace{-0cm}\hfill\includegraphics[width=2.5cm]{../config/logo_cif} 
  \includegraphics[width=1.9cm]{../config/seal_wwz} 
  \\ \vspace{2em}
  \usebeamerfont{title}\textbf{\inserttitle}\par
  \usebeamerfont{title}\usebeamercolor[fg]{title}\insertsubtitle\par  \vspace{1.5em}
  \small\usebeamerfont{author}\insertauthor\par
  \usebeamerfont{author}\insertinstitute\par \vspace{2em}
  \usebeamercolor[fg]{titlegraphic}\inserttitlegraphic
    \tiny \noindent \texttt{Commit Hash: \gitHash}\\ 
	\texttt{Commit Time: \gitAuthorIsoDate}\\ \vspace{1em}
  \link \href{https://github.com/cifunibas/Bitcoin-Blockchain-Cryptoassets/blob/main/slides/intro.pdf}
  {Get most recent version}\\
  \link \href{https://github.com/cifunibas/Bitcoin-Blockchain-Cryptoassets/blob/main/slides/intro.pdf}
  {Watch video lecture}\\ \vspace{1em}
  License: \texttt{Creative Commons Attribution-NonCommercial-ShareAlike 4.0 International}\\\vspace{2em}
  \includegraphics[width = 1.2cm]{../config/license}
}


%%%%%%%%%%%%%%%%%%%%%%%%%%%%%%%%%%%%%%%%%%%%%%
%%%%%%%%%%%%%%%%%%%%%%%%%%%%%%%%%%%%%%%%%%%%%%
\begin{document}

\thispagestyle{empty}
\begin{frame}[noframenumbering]
	\titlepage
\end{frame}


%%%
\begin{frame}{Quantifying Illicit Activity}
	Illicit activities with cryptocurrencies pose a certain problem but are hard to quantify. \\
	\vspace{1em}
	Number of users or transactions are flawed measurables:
		\begin{itemize}
			\item One user $\rightarrow$ Multiple addresses
			\item Multiple Users $\rightarrow$ One address
			\item Transaction $\neq$ Transaction
			\item Obfuscation transactions
		\end{itemize}
	\vspace{1em} 
	Studies use different assumptions in order to carry out estimates, which can have a big influence on the results.	
\end{frame}
%%%	


%%%
\begin{frame}{Origin and Homogenity of Bitcoin Units}
	Each output has a clearly distinguishable origin.
	\begin{figure}
		\resizebox{10cm}{6cm}{
			\begin{tikzpicture}[
     		 roundnode1/.style = {circle,  draw=highlight, fill=highlight!5},
     		 roundnode2/.style = {circle,  draw=focus!50, fill=focus!5},
      		squarednode/.style = {rectangle, draw=black!60, fill=black!5},
     		 ]
			
    \node[roundnode2]    (nodeA)                                    {\texttt{A}};
    \node[roundnode2]    (nodeB)        [below=8mm of nodeA]        {\texttt{B}};
    \node[roundnode2]    (nodeC)        [below=8mm of nodeB]        {\texttt{C}};
    \node[roundnode2]    (nodeD)        [below=8mm of nodeC]        {\includegraphics[scale=0.025]{../assets/images/agents/intermediary_devil}};
    \node[roundnode2]    (nodeE)        [below=10mm of nodeD]       {\texttt{E}};
    
    \node[squarednode]  (TRX1)          [right =of nodeA]           {\texttt{TRX1}}; 
    \node[squarednode]  (TRX2)          [below =17mm of TRX1]       {\texttt{TRX2}};
    \node[squarednode]  (TRX3)          [right =8mm of nodeD]      {\texttt{TRX3}};
    \node[squarednode]  (TRX4)          [right =of nodeE]           {\texttt{TRX4}};
    
    \node[roundnode2]    (nodeF)        [right =of TRX1]            {\texttt{F}};
    \node[roundnode2]    (nodeG)        [right =of TRX2]            {\texttt{G}};
    \node[roundnode2]    (nodeI)        [right =of TRX3]            {\texttt{I}};
    \node[roundnode2]    (nodeH)        [above =4mm of nodeI]       {\texttt{H}};
    \node[roundnode2]    (nodeJ)        [below =2mm of nodeI]       {\texttt{J}};
    \node[roundnode2]    (nodeK)        [right =of TRX4]            {\texttt{K}};
    
    \node[squarednode]  (TRX5)          [right =of nodeF]          {\texttt{TRX5}}; 
    \node[squarednode]  (TRX6)          [right =of nodeG]          {\texttt{TRX6}};
    \node[squarednode]  (TRX7)          [right =of nodeK]          {\texttt{TRX7}};
    
    \node[roundnode1]    (nodeO)        [right =of TRX6]            {\texttt{O}};
    \node[roundnode2]    (nodeN)        [above =3mm of nodeO]       {\texttt{N}};
    \node[roundnode2]    (nodeM)        [above =3mm of nodeN]       {\texttt{M}};
    \node[roundnode1]    (nodeL)        [above =1mm of nodeM]       {\texttt{L}};
    \node[roundnode2]    (nodeP)        [below =4mm of nodeO]       {\texttt{P}};
    \node[roundnode2]    (nodeQ)        [right =of TRX7]            {\texttt{Q}};
    
    \node[squarednode]  (TRX9)          [right = 48mm of nodeI]    {\texttt{TRX9}};
    \node[squarednode]  (TRX8)          [above = 33mm of TRX9]     {\texttt{TRX8}};
    
    \node[roundnode1]    (nodeS)        [right =of TRX8]            {\texttt{S}};
    \node[roundnode1]    (nodeR)        [above =2mm of nodeS]       {\texttt{R}};
    \node[roundnode1]    (nodeT)        [below =2mm of nodeS]       {\texttt{T}};
    \node[roundnode1]    (nodeU)        [right =of TRX9]            {\texttt{U}};
    
    
    \draw[-] (nodeA) -- (TRX1);
    \draw[-] (nodeB) -- (TRX2);
    \draw[-] (nodeC) -- (TRX2);
    \draw[-] (nodeD) -- (TRX3);
    \draw[-] (nodeE) -- (TRX4);
    
    \draw[-] (TRX1) -- (nodeF);
    \draw[-] (TRX2) -- (nodeG);
    \draw[-] (TRX3) -- (nodeH);
    \draw[-] (TRX3) -- (nodeI);
    \draw[-] (TRX3) -- (nodeJ);
    \draw[-] (TRX4) -- (nodeK);
    
    \draw[-] (nodeF) -- (TRX5);
    \draw[-] (nodeG) -- (TRX6);
    \draw[-] (nodeH) -- (TRX6);
    \draw[-] (nodeJ) -- (TRX7);
    \draw[-] (nodeK) -- (TRX7);
    
    \draw[-] (TRX5) -- (nodeL);
    \draw[-] (TRX5) -- (nodeM);
    \draw[-] (TRX6) -- (nodeN);
    \draw[-] (TRX6) -- (nodeO);
    \draw[-] (TRX6) -- (nodeP);
    \draw[-] (TRX7) -- (nodeQ);
    
    \draw[-] (nodeM) -- (TRX8);
    \draw[-] (nodeN) -- (TRX8);
    \draw[-] (nodeP) -- (TRX9);
    \draw[-] (nodeQ) -- (TRX9);
    \draw[-] (nodeI) -- (TRX9);
    
    \draw[-] (TRX8) -- (nodeR);
    \draw[-] (TRX8) -- (nodeS);
    \draw[-] (TRX8) -- (nodeT);
    \draw[-] (TRX9) -- (nodeU);
 
			\end{tikzpicture}
		}
	\end{figure}	
\end{frame}
%%%


%%%
\begin{frame}{Silk Road}
	\begin{itemize}
		\item First modern large-scale darknet market
		\item Trading of illegal drugs and digital goods
		\item Bitcoin as dominant medium of exchange
	\end{itemize}
	\vspace{1em}
	\centering
	\begin{tikzpicture}[squarednode/.style = {rectangle, draw=black!60, fill=black!5}]
		\node (AgentSeller)				{\includegraphics[scale=0.05]{../assets/images/agents/handing_right}};
		\node (Seller)	[below= 0.05cm of AgentSeller]			{Seller};
	
		\node (Darknet)          [right = 2cm of AgentSeller]          {\includegraphics[scale=0.1]{../assets/images/darknet}}; 
		\node (Silkroad)	[below= 0.05cm of Darknet]			{Silkroad};
	
		\node (AgentBuyer)		[right =2cm of Darknet]		{\includegraphics[scale=0.05]{../assets/images/agents/handing_money_left}};
		\node (Buyer)	[below= 0.05cm of AgentBuyer]			{Buyer};
	
		\draw[->, thick] (AgentBuyer) edge [out=-230, in=50] node[midway,above] {\texttt{BTC}} (Darknet);
		\draw[->, thick, dotted] (Darknet) edge [out=-230, in=50] node[midway,above] {\texttt{BTC}} (AgentSeller);
		\draw[->, thick] (AgentSeller) edge [out=-45, in=-140] node[midway,below] {\texttt{Good}} (AgentBuyer);
	\end{tikzpicture}
\end{frame}
%%%


%%%
\begin{frame}{Mt. Gox}
	\centering
		\begin{itemize}
			\item Worlds largest bitcoin exchange in 2013.
			\item Transaction malleability as reason for stopping Bitcoin withdrawals in February 2014.
			\item Mistake: Relied solely on the transaction hash to track and verify its account balance.
			\item Claim that transaction malleability as the reason for the loss of around 850,000 BTC is controversial. See \cite{Decker2014}
		\end{itemize}
	\includegraphics[scale=0.12]{../assets/images/mt_gox}\\
	\footnotesize{Picture source: Wikipedia}	
\end{frame}
%%%


%%%
\begin{frame}{Malleability Attack}
	\centering
	\begin{tikzpicture}[squarednode/.style = {rectangle, draw=black!60, fill=black!5}]
		%User
		\node (AvatarUser) at (0,0)	{\includegraphics[scale=0.05]{../assets/images/agents/agent_right}};
		\node (User)[below= 0.05cm of AvatarUser]{{\footnotesize User}};
		
%Mt.Gox
		\node (CEX)	[right =3cm of AvatarUser]{\includegraphics[scale=0.05]{../assets/images/agents/handing_money_left}};
		\node (Mt.Gox)[below= 0.05cm of CEX]{{\footnotesize Mt.Gox}};
		
%Connection
	\only<1->{
		\draw[->, thick, dotted](AvatarUser) edge [out=-30, in=-150] node[midway,below] {{\scriptsize Withdrawal Request}} (CEX);
		}
	\only<2->{	
		\draw[->, thick, dotted] (CEX) edge [out=-210, in=30] node[midway,above] {{\scriptsize $TXID_{a}$}} (AvatarUser);
		}
		
%Network nodes	
	\node (agenta) at (-1.5,1) {\includegraphics[width = 0.6 cm]{../assets/images/agents/avatar_rand3.png}};
	\node (agentb) at (-1.5,0) {\includegraphics[width = 0.6 cm]{../assets/images/agents/avatar_rand4.png}};
	\node (agentc) at (-1.5,-1) {\includegraphics[width = 0.6 cm]{../assets/images/agents/avatar_rand5.png}};
	\node (agentd) at (5.8,0.5) {\includegraphics[width = 0.6 cm]{../assets/images/agents/avatar_rand1.png}};
	\node (agente) at (5.8,-0.5) {\includegraphics[width = 0.6 cm]{../assets/images/agents/avatar_rand2.png}};	

%Peer connections
	\only<3->{
	\draw[->, thick, dotted]	(AvatarUser.north west) -- (agenta.east) node[midway, above] {\scriptsize $TXID_{b}$};
	\draw[->, thick, dotted] 	(AvatarUser.west) -- (agentb.east);
	\draw[->, thick, dotted]	(AvatarUser.south west) -- (agentc.east);
		}
		
\only<2->{
	\draw[->, thick, dotted]	(CEX.east) -- (agentd.west)  node[midway, above= 3mm] {\scriptsize $TXID_{a}$};
	\draw[->, thick, dotted] 	(CEX.east) -- (agente.west)  ;
		}
	\end{tikzpicture}
	\begin{itemize}	
		\item<3->[1.] The user sends a request for withdrawal. The exchange initiates $TRX_a$ with $TXID_a$.
		\item<4->[2.] The user modifies $TRX_a$'s \texttt{scriptSig} in a way that the transaction is still valid, but its ID changes to $TXID_b$. Both transactions $TRX_a$ and $TRX_b$ are valid and race for confirmation.
		\item<5->[3.] If the modified version $TRX_b$ gets included in the blockchain:
		\begin{itemize}
			\item The user receives the expected BTC units through $TRX_b$.
			\item $TRX_a$ fails and the user is still credited with funds in Mt. Gox's system.
		\end{itemize}
	\end{itemize}
\end{frame}
%%%


%%%
\begin{frame}{Wannacry}
	\centering
		\includegraphics[scale=0.28]{../assets/images/wannacry} \\
		\footnotesize{Picture source: OneSpan Blog}\\
		\vspace{1em}
		\begin{small}
			\href{https://blockstream.info/address/13AM4VW2dhxYgXeQepoHkHSQuy6NgaEb94}{\texttt{13AM4VW2dhxYgXeQepoHkHSQuy6NgaEb94} \link} \\
			\href{https://blockstream.info/address/12t9YDPgwueZ9NyMgw519p7AA8isjr6SMw}{\texttt{12t9YDPgwueZ9NyMgw519p7AA8isjr6SMw} \link} \\
			\href{https://blockstream.info/address/115p7UMMngoj1pMvkpHijcRdfJNXj6LrLn} {\texttt{115p7UMMngoj1pMvkpHijcRdfJNXj6LrLn} \link} \\
		\end{small}
\end{frame}
%%%


%%%
\begin{frame}{Other Risks \& Illicit Activities}
	\textbf{Botnet Miner}
		\begin{itemize}
			\item Malware that integrates a victims computer into the "botnet".
			\item Compromised computers can be used for mining.
		\begin{center}
		\includegraphics[scale=0.2]{../assets/images/microsoft_store}\\
		\footnotesize{Picture source: Symantec}
		\end{center}
		\end{itemize}
	%\vspace{1em}
	\textbf{Bitcoin Tumbler}
		\begin{itemize}
			\item Used to disguise the origin of Bitcoin units and links between old and new addresses.
			\item How: Send coins from users around, Randomize transaction amounts, Add time delays
		\end{itemize}	
\end{frame}
%%%


%%%
\begin{frame}{Regulation}
	Bitcoin Network because of decentralized nature hard to regulate $\rightarrow$ Focus on On- and Off-ramps\\
	\vspace{1em}
	Example: {\color{focus} OpenVASP } (Virtual asset service providers)
		\begin{itemize}
			\item Protocol facilitating compliance with global travel rule requirements for VASPs. Shared communication protocol to exchange VA transfer information.
		\end{itemize}
	\vspace{1em}
	\textbf{Because of the high transparency, Bitcoin is not very suitable for usage with illegal activities.}
\end{frame}
%%%

\begin{frame}%[allowframebreaks]
\frametitle{References and Recommended Reading}
	\bibliographystyle{amsplain}
	\bibliography{../assets/bib/refs}
\end{frame}


\end{document}
