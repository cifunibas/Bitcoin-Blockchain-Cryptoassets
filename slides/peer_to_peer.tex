% Choose one to switch between slides and handout
%\documentclass[]{beamer}
\documentclass[handout]{beamer}

% Video Meta Data
\title{Bitcoin, Blockchain and Cryptoassets}
\subtitle{Peer-to-Peer Networks}
\author{Prof. Dr. Fabian Schär}
\institute{University of Basel}

% Config File
% Packages
\usepackage[utf8]{inputenc} 
\usepackage{hyperref}
\usepackage{gitinfo2}
\usepackage{tikz}
\usepackage{amsmath}
\usepackage{bibentry}
\usepackage{xcolor}
\usepackage{caption}

% Beamer Template Options
\beamertemplatenavigationsymbolsempty
\setbeamertemplate{footline}[frame number]
\setbeamercolor{structure}{fg=black}
\setbeamercolor{footline}{fg=black}
\setbeamercolor{title}{fg=black}
\setbeamercolor{frametitle}{fg=black}
\setbeamercolor{item}{fg=black}
\setbeamercolor{}{fg=black}
\setbeamercolor{bibliography item}{fg=black}
\setbeamercolor*{bibliography entry title}{fg=black}
\setbeamertemplate{items}[square]
\setbeamertemplate{enumerate items}[default]
\captionsetup[figure]{labelfont={color=black},font={color=black}}
\captionsetup[table]{labelfont={color=black},font={color=black}}

\setbeamertemplate{bibliography item}{\insertbiblabel}

% Link Icon Command
\newcommand{\link}{%
    \tikz[x=1.2ex, y=1.2ex, baseline=-0.05ex]{% 
        \begin{scope}[x=1ex, y=1ex]
            \clip (-0.1,-0.1) 
                --++ (-0, 1.2) 
                --++ (0.6, 0) 
                --++ (0, -0.6) 
                --++ (0.6, 0) 
                --++ (0, -1);
            \path[draw, 
                line width = 0.5, 
                rounded corners=0.5] 
                (0,0) rectangle (1,1);
        \end{scope}
        \path[draw, line width = 0.5] (0.5, 0.5) 
            -- (1, 1);
        \path[draw, line width = 0.5] (0.6, 1) 
            -- (1, 1) -- (1, 0.6);
        }
    }

% Custom Titlepage
\defbeamertemplate*{title page}{customized}[1][]
{
  \vspace{-0cm}\hfill\includegraphics[width=2.5cm]{../config/logo_cif} 
  \includegraphics[width=1.9cm]{../config/seal_wwz} 
  \\ \vspace{2em}
  \usebeamerfont{title}\textbf{\inserttitle}\par
  \usebeamerfont{title}\usebeamercolor[fg]{title}\insertsubtitle\par  \vspace{1.5em}
  \small\usebeamerfont{author}\insertauthor\par
  \usebeamerfont{author}\insertinstitute\par \vspace{2em}
  \usebeamercolor[fg]{titlegraphic}\inserttitlegraphic
    \tiny \noindent \texttt{Commit Hash: \gitHash}\\ 
	\texttt{Commit Time: \gitAuthorIsoDate}\\ \vspace{1em}
  \link \href{https://github.com/cifunibas/Bitcoin-Blockchain-Cryptoassets/blob/main/slides/intro.pdf}
  {Get most recent version}\\
  \link \href{https://github.com/cifunibas/Bitcoin-Blockchain-Cryptoassets/blob/main/slides/intro.pdf}
  {Watch video lecture}\\ \vspace{1em}
  License: \texttt{Creative Commons Attribution-NonCommercial-ShareAlike 4.0 International}\\\vspace{2em}
  \includegraphics[width = 1.2cm]{../config/license}
}


%%%%%%%%%%%%%%%%%%%%%%%%%%%%%%%%%%%%%%%%%%%%%%
%%%%%%%%%%%%%%%%%%%%%%%%%%%%%%%%%%%%%%%%%%%%%%

\begin{document}

\thispagestyle{empty}
\begin{frame}[noframenumbering]
	\titlepage
\end{frame}

%%%
\begin{frame}{The Origin of Peer-to-Peer Networks}
	Peer-to-peer (p2p) networks were popularized by file sharing systems in the early 2000s.
	\begin{columns}
		\begin{column}{0.5\textwidth}
			\begin{figure}
				\begin{center}
					\includegraphics[height = 0.25\textheight]{../assets/images/napster}
				\end{center}
			\end{figure}
		\end{column}
		\begin{column}{0.5\textwidth}
			\begin{figure}
				\begin{center}
					\includegraphics[height = 0.17\textheight]{../assets/images/btt}
				\end{center}
			\end{figure}
		\end{column}
	\end{columns}
	\begin{columns}
		\begin{column}{0.5\textwidth}
			\begin{center}
				Napster
			\end{center}
		\end{column}
		\begin{column}{0.5\textwidth}
			\begin{center}
				BitTorrent
			\end{center}
		\end{column}
	\end{columns}
	\vspace{1.8em}
	\uncover<2->{They allow network participants to directly interact with eachother, without the need for servers or stable hosts.}
\end{frame}
%%%

%%%
\begin{frame}{Peer-to-Peer vs. Centralized Networks}
	\vspace{1em}
	In a peer-to-peer network, each participant...
	\vspace{0.5em}
%	\begin{footnotesize}
		\begin{itemize}
			\item<2-> Can connect to any number of other participants
			\item<3-> Has equal privileges (no administration)
			\item<4-> Is consumer and provider of resources (client and server)
		\end{itemize}
%	\end{footnotesize}
	\begin{columns}[T]
		\begin{column}{0.5\textwidth}
			\vspace{2.5em}
			\begin{figure}
				
% Title
%\node[above] at (2,4.3) {Peer-to-Peer};


% Network
\node (agenta) at (1,2.8) {\includegraphics[width = 0.6 cm]{../assets/images/agents/avatar_rand3.png}};
\node (agentb) at (0.5,1) {\includegraphics[width = 0.6 cm]{../assets/images/agents/avatar_rand4.png}};
\node (agentc) at (3,2.1) {\includegraphics[width = 0.6 cm]{../assets/images/agents/avatar_rand5.png}};
\node (agentd) at (2.8,0) {\includegraphics[width = 0.6 cm]{../assets/images/agents/avatar_rand1.png}};
\node (agente) at (5,4.3) {\includegraphics[width = 0.6 cm]{../assets/images/agents/avatar_rand2.png}};	
\node (agentf) at (5.1,1.1) {\includegraphics[width = 0.6 cm]{../assets/images/agents/avatar_rand3.png}};
\node (agentg) at (7.5,3.8) {\includegraphics[width = 0.6 cm]{../assets/images/agents/avatar_rand4.png}};
\node (agenth) at (6.7,0.4) {\includegraphics[width = 0.6 cm]{../assets/images/agents/avatar_rand5.png}};

% Network flow
\draw[<->, thick, dashed]	(agenta.south) -- (agentb.north);
\draw[<->, thick, dashed] 	(agenta.east) -- (agente.west);
\draw[<->, thick, dashed]	(agenta.south east) -- (agentc.west);
\draw[<->, thick, dashed]	(agente.south west) -- (agentc.north east);
\draw[<->, thick, dashed]	(agente.south) -- (agentf.north);
\draw[<->, thick, dashed]	(agente.east) -- (agentg.west);
\draw[<->, thick, dashed]	(agentc.south west) -- (agentb.east);
\draw[<->, thick, dashed]	(agentc.south) -- (agentd.north);
\draw[<->, thick, dashed]	(agentc.south east) --  (agentf.west);
\draw[<->, thick, dashed]	(agentg.south west) -- (agentf.north east);
\draw[<->, thick, dashed]	(agentg.south) -- (agenth.north);
\draw[<->, thick, dashed]	(agentb.south east) -- (agentd.west);
\draw[<->, thick, dashed]	(agentf.south west) -- (agentd.east);
\draw[<->, thick, dashed]	(agentf.south east) -- (agenth.west);
\draw[<->, thick, dashed]	(agenth.south west) -- (agentd.east);

			\end{figure}
		\end{column}
		\begin{column}{0.5\textwidth}
			\begin{figure}
				
% Title
%\node[above] at (2.5,5.7) {Centralized};


% Network
\node (server) at (4.5,2.8) {\includegraphics[width = 0.5 cm]{../assets/images/agents/intermediary.png}};
\node (agenta) at (1,4.7) {\includegraphics[width = 0.6 cm]{../assets/images/agents/avatar_rand3.png}};
\node (agentb) at (0.5,2) {\includegraphics[width = 0.6 cm]{../assets/images/agents/avatar_rand4.png}};
\node (agentc) at (2.8,0) {\includegraphics[width = 0.6 cm]{../assets/images/agents/avatar_rand1.png}};
\node (agentd) at (5,5.7) {\includegraphics[width = 0.6 cm]{../assets/images/agents/avatar_rand2.png}};	
\node (agente) at (7.5,3.8) {\includegraphics[width = 0.6 cm]{../assets/images/agents/avatar_rand4.png}};
\node (agentf) at (6.7,0.2) {\includegraphics[width = 0.6 cm]{../assets/images/agents/avatar_rand5.png}};

% Network flow
\draw[<->, thick, dashed]	(agenta.east) -- (server.north west);
\draw[<->, thick, dashed] 	(agentb.east) -- (server.west);
\draw[<->, thick, dashed]	(agentc.north east) -- (server.south west);
\draw[<->, thick, dashed]	(agentd.south) -- (server.north);
\draw[<->, thick, dashed]	(agente.west) -- (server.east);
\draw[<->, thick, dashed]	(agentf.north west) -- (server.south east);


			\end{figure}
		\end{column}
	\end{columns}
\end{frame}
%%%

%%%
\begin{frame}{Node Software}
	To facilitate a peer-to-peer network, each participant installs a network specific piece of software which establishes them as a node in the network and is able to:
	\begin{itemize}
		\item Discover other nodes
		\item Establish connections to discovered nodes
	\end{itemize}
	\vspace{1em}
	Furthermore, in a pure p2p network, each node acts as a client and server simultaneously. The node software should therefore also be able to:
	\begin{itemize}
		\item Share local resources (e.g., storage or computing power)
		\item Access shared resources
	\end{itemize}
\end{frame}
%%%

%%%
\begin{frame}{Network Specification}
	Each peer-to-peer network formally describes the communication between nodes in a \color{focus}protocol specification document\color{black}. Node software clients then implement and adhere to these specifications.

	\vspace{1em}

	This makes it possible for nodes in a peer-to-peer network to run with different node software, \color{focus}further decentralizing \color{black}the system.
\end{frame}
%%%

%%%
\begin{frame}{Peer Discovery}
	P2p networks can be grouped into two categories: \color{focus}Unstructured and structured networks\color{black}. The latter are advantageous when the shared resources are heterogeneous. They provide a way to quickly find even very rare resources in the network. However, this comes at the cost of reduced decentralization.
	
	\vspace{1em}
	
	For a p2p payment network, the resources on the nodes are similar or even identical. Consequently, \color{focus}most existing p2p payment networks choose an unstructured approach\color{black}.
	
	\vspace{1em}
	
	Steps to connect to peers in an unstructured p2p network:
	\begin{enumerate}
		\item Connect to a random node in a seed list
		\item Ask this node for a list of their connections
		\item Connect to a random set of nodes on this list
		\item Repeat steps 2 and 3 until enough connections are established
	\end{enumerate}
\end{frame}
%%%

%%%
\begin{frame}{Advantages of P2P Networks}
	Due to its architecture, a p2p network offers key advantages over a centralized network:
	\begin{itemize}
		\item Easy to scale
			\begin{itemize}
				\item No bandwidth bottlenecks
				\item Only requires connection to a subset of peers
			\end{itemize}
		\item No central point for an attack
			\begin{itemize}
				\item No single node is critical for a functioning network
				\item No node has special privileges or powers
				\item Very difficult to overload the network via DoS attacks
			\end{itemize}	
		\item Every participant supplies hardware
		\item Difficult to regulate or stop
	\end{itemize}
	
\end{frame}
%%%

%%%
\begin{frame}{Problems of P2P Networks}
	While p2p networks provide a highly decentralized system, there are trade-offs, especially regarding security:
	\begin{itemize}
		\item A node can share malicious, illegal or irrelevant resources
		\item Difficult to regulate illicit activity
		\item The network can be limited by the nodes with the weakest hardware
	\end{itemize}
	
\end{frame}
%%%


\begin{frame}%[allowframebreaks]
\frametitle{References}
	\bibliographystyle{amsplain}
	\bibliography{../assets/bib/refs}
\end{frame}

\end{document}