% Choose one to switch betweeen slides and handout
%\documentclass[]{beamer}
\documentclass[handout]{beamer}

% Video Meta Data
\title{Bitcoin, Blockchain and Cryptoassets}
\subtitle{Bitcoin Applications}
\author{Prof. Dr. Fabian Schär}
\institute{University of Basel}

% Config File
% Packages
\usepackage[utf8]{inputenc} 
\usepackage{hyperref}
\usepackage{gitinfo2}
\usepackage{tikz}
\usepackage{amsmath}
\usepackage{bibentry}
\usepackage{xcolor}
\usepackage{caption}

% Beamer Template Options
\beamertemplatenavigationsymbolsempty
\setbeamertemplate{footline}[frame number]
\setbeamercolor{structure}{fg=black}
\setbeamercolor{footline}{fg=black}
\setbeamercolor{title}{fg=black}
\setbeamercolor{frametitle}{fg=black}
\setbeamercolor{item}{fg=black}
\setbeamercolor{}{fg=black}
\setbeamercolor{bibliography item}{fg=black}
\setbeamercolor*{bibliography entry title}{fg=black}
\setbeamertemplate{items}[square]
\setbeamertemplate{enumerate items}[default]
\captionsetup[figure]{labelfont={color=black},font={color=black}}
\captionsetup[table]{labelfont={color=black},font={color=black}}

\setbeamertemplate{bibliography item}{\insertbiblabel}

% Link Icon Command
\newcommand{\link}{%
    \tikz[x=1.2ex, y=1.2ex, baseline=-0.05ex]{% 
        \begin{scope}[x=1ex, y=1ex]
            \clip (-0.1,-0.1) 
                --++ (-0, 1.2) 
                --++ (0.6, 0) 
                --++ (0, -0.6) 
                --++ (0.6, 0) 
                --++ (0, -1);
            \path[draw, 
                line width = 0.5, 
                rounded corners=0.5] 
                (0,0) rectangle (1,1);
        \end{scope}
        \path[draw, line width = 0.5] (0.5, 0.5) 
            -- (1, 1);
        \path[draw, line width = 0.5] (0.6, 1) 
            -- (1, 1) -- (1, 0.6);
        }
    }

% Custom Titlepage
\defbeamertemplate*{title page}{customized}[1][]
{
  \vspace{-0cm}\hfill\includegraphics[width=2.5cm]{../config/logo_cif} 
  \includegraphics[width=1.9cm]{../config/seal_wwz} 
  \\ \vspace{2em}
  \usebeamerfont{title}\textbf{\inserttitle}\par
  \usebeamerfont{title}\usebeamercolor[fg]{title}\insertsubtitle\par  \vspace{1.5em}
  \small\usebeamerfont{author}\insertauthor\par
  \usebeamerfont{author}\insertinstitute\par \vspace{2em}
  \usebeamercolor[fg]{titlegraphic}\inserttitlegraphic
    \tiny \noindent \texttt{Commit Hash: \gitHash}\\ 
	\texttt{Commit Time: \gitAuthorIsoDate}\\ \vspace{1em}
  \link \href{https://github.com/cifunibas/Bitcoin-Blockchain-Cryptoassets/blob/main/slides/intro.pdf}
  {Get most recent version}\\
  \link \href{https://github.com/cifunibas/Bitcoin-Blockchain-Cryptoassets/blob/main/slides/intro.pdf}
  {Watch video lecture}\\ \vspace{1em}
  License: \texttt{Creative Commons Attribution-NonCommercial-ShareAlike 4.0 International}\\\vspace{2em}
  \includegraphics[width = 1.2cm]{../config/license}
}


% Adiitional Files
\input{../config/solidity.tex} % specification of solidity color coding
\input{../config/json.tex} % specification of JSON color coding
% Defining Bitcoin Symbol
\def\btc{%
	\leavevmode
	\vtop{\offinterlineskip %\bfseries
		\setbox0=\hbox{B}%
		\setbox2=\hbox to\wd0{\hfil\hskip-.03em
			\vrule height .3ex width .15ex\hskip .08em
			\vrule height .3ex width .15ex\hfil}
		\vbox{\copy2\box0}\box2}} % definition of bitcoin symbol


%%%%%%%%%%%%%%%%%%%%%%%%%%%%%%%%%%%%%%%%%%%%%%
%%%%%%%%%%%%%%%%%%%%%%%%%%%%%%%%%%%%%%%%%%%%%%
\begin{document}

\thispagestyle{empty}
\begin{frame}[noframenumbering]
	\titlepage
\end{frame}

%%%

\begin{frame}{External Data}	
	\begin{itemize}
		\item<1 ->\textbf{Native On-Chain Data}
		\begin{itemize}
			\item<1 ->Data that is stored on-chain and fully secured by the consensus protocol (e.g. Bitcoin trx)\\
			\vspace{0.25em}$\rightarrow$ Automated evaluation and verification
		\end{itemize}
		\vspace{1em}
		\item<2 ->\textbf{Added Off-Chain Data}
		\begin{itemize}
			\item<2 -> Data that is stored on-chain but not secured by the consensus protocol (e.g. course certificates)\\
			\vspace{0.25em}$\rightarrow$ Automated evaluation, subject to data quality
		\end{itemize}
		\vspace{1em}
		\item<3 -> \textbf{Off-Chain Data}
		\begin{itemize}
			\item<3 -> Data that is NOT directly stored on the Blockchain (e.g. weather, results of a football game)\\
			\vspace{0.25em}$\rightarrow$ Requires trustworthy data providers (oracles)
		\end{itemize}
	\end{itemize}
\end{frame}

%%%

\begin{frame}{External Data and the Oracle Poblem}
	\begin{columns}
		\begin{column}{0.6\textwidth}
			\begin{itemize}[<+->]
				\item Example: Tracking ownership of a parcel
				\item On-Chain representation of parcel with unique token
				\item Reproducible crypto-anker (e.g. QR-code) is problematic
			\end{itemize}
		\end{column}
		\begin{column}{0.4\textwidth}
			\includegraphics[width=4cm]{../assets/images/parcel.png}
		\end{column}
	\end{columns}
\end{frame}

%%%

\begin{frame}{Oracles on Bitcoin}
	\textbf{Example:} Alice and Bob want to bet on some event.\\
	\vspace{0.25cm}
	\uncover<2 ->{\textbf{Three possibilities:}}
	\begin{itemize}
		\vspace{0.1cm}
		\item<2->[1.] Both parties transfer their stake to the address of the oracle. The oracle then sends the prize to the winning party after the event has taken place.
		\begin{itemize}
			\item<2-> Problem: Control over funds lies solely with the oracle.
		\end{itemize}
		\vspace{0.2cm}
		\item<3->[2.] They create a 2-of-3 multisig transaction. Alice, Bob and the oracle each hold a key.
		\begin{itemize}
			\item<3-> Control distributed and Alice and Bob can overrule the oracle.
		\end{itemize}
		\vspace{0.2cm}
		\item<4->[3.] More than one oracle and use m-of-n multisig transaction. 
		\begin{itemize}
			\item<4-> Betting parties hold $\frac{m}{2}$ keys and \small{$m-1$} oracles hold one key.
			\item<4-> Any $m-of-2m-1$ multisig implementation possible ($m$ has to be an even integer).
		\end{itemize}
	\end{itemize}
\end{frame}

%%%

\begin{frame}{Proof of Existence}
	\centering
	\begin{tikzpicture} 
		\uncover<+->{\node (img2) {\includegraphics[width = 10cm, frame]{../assets/images/google1.PNG}};}
		\uncover<+->{\node (img3) at (img2) {\includegraphics[width = 10cm, frame]{../assets/images/google2.PNG}};}
		\uncover<+->{\node (img4) at (img2) {\includegraphics[width = 10cm, frame]{../assets/images/google3.PNG}};}
		\uncover<+->{\node (img5) {\includegraphics[width = 10cm, frame]{../assets/images/google4.PNG}};}
		\uncover<+->{\node (img6) at (img2) {\includegraphics[width = 10cm, frame]{../assets/images/google5.PNG}};}
		\uncover<+->{\node (img7) at (img2) {\includegraphics[width = 10cm, frame]{../assets/images/google6.PNG}};}
	\end{tikzpicture}
\end{frame}

%%%
\begin{frame}{Diplomas on the Blockchain}
	\begin{minipage}{\linewidth}
		\textbf{\large{Issuance:}}
			\begin{columns}
			\begin{column}{0.3\textwidth}
				\centering
				\includegraphics[width=1.5cm]{../assets/images/diploma.png}\\
				University issues\\
				diploma
				\end{column}
			\begin{column}{0.05\textwidth}
				\includegraphics[width=0.75cm]{../assets/images/big_arrow.png}
			\end{column}
			\begin{column}{0.3\textwidth}
				\centering
				\includegraphics[width=1.5cm]{../assets/images/fingerprint.png}\\
				Computes hash value of diploma
			\end{column}
			\begin{column}{0.05\textwidth}
				\includegraphics[width=0.75cm]{../assets/images/big_arrow.png}
			\end{column}
			\begin{column}{0.3\textwidth}
				\centering
				\includegraphics[width=1.5cm]{../assets/images/write_onchain.png}\\
				Saves hash value\\ on blockchain
			\end{column}
		\end{columns}
	\end{minipage}\vfill
	\vspace{0.5cm}
	\begin{minipage}{\linewidth}
	\textbf{\large{Verification:}}
	\begin{columns}
		\begin{column}{0.3\textwidth}
			\centering
			\includegraphics[width=1.5cm]{../assets/images/receive_diploma.png}\\
			Potential employer receives diploma
		\end{column}
		\begin{column}{0.05\textwidth}
			\includegraphics[width=0.75cm]{../assets/images/big_arrow.png}
		\end{column}
		\begin{column}{0.3\textwidth}
			\centering
			\includegraphics[width=1.5cm]{../assets/images/fingerprint.png}\\
			Computes hash value of diploma
		\end{column}
		\begin{column}{0.05\textwidth}
			\includegraphics[width=0.75cm]{../assets/images/big_arrow.png}
		\end{column}
		\begin{column}{0.3\textwidth}
			\centering
			\includegraphics[width=1.5cm]{../assets/images/verify_onchain.png}\\
			Compares with on-chain hash value
		\end{column}
	\end{columns}
\end{minipage}
\end{frame}

%%%

\begin{frame}{Proof of Existence}
	\begin{columns}
		\begin{column}{0.4\textwidth}
			\begin{figure}
				\centering
				\includegraphics[width=4cm, frame]{../assets/images/Diploma_unibas.PNG}
				\scriptsize{\href{https://cif.unibas.ch/en/events-projects/certificates/}{cif.unibas.ch/en/}}
			\end{figure}
		\end{column}
		\begin{column}{0.6\textwidth}
			\begin{itemize}
				\item<2 -> Certificate for Blockchain-related courses since 2018 
				\item<3 -> Roll-out for entire University of Basel coming soon
			\end{itemize}
		\end{column}
	\end{columns}
\end{frame}

%%%

\begin{frame}{Commit-and-reveal}
	All information on a public blockchain is, well, public.\\
	Two individuals\\
	Show this on blocks. Rock paper scissor
\end{frame}

%%%

\begin{frame}{Commit-and-reveal}
 Second page needed?
\end{frame}

%%%

\begin{frame}[fragile]{Merkle Proofs}
	\begin{figure}
		\begin{minipage}[h]{0.55\linewidth}
			\centering
			\includegraphics[width=4cm]{../assets/images/id.PNG}
		\end{minipage}%
		\hfill
		\begin{minipage}[h]{0.45\linewidth}
			\begin{itemize}
				\item<1 ->Alice has to proof a certain age
				\item<1 ->But Alice does not want to disclose other private information
			\end{itemize}
		\end{minipage}
	\end{figure}
	\vspace{-0.5cm}
	\begin{figure}
		\begin{minipage}[h]{0.55\linewidth}
			\begin{lstlisting}[language=json,firstnumber=1]
				{
					"Name": "Alice",
					"SaltName": 103982023492,
					"DateOfBirth": "1/1/2000",
					"SaltDate": 787793837929,
					"PlaceOfBirth": "Basel",
					"SaltPlace": 989229104925
				}
			\end{lstlisting}
			\centering
			\scriptsize{Passport as a JSON object}
		\end{minipage}%
		\hfill
		\begin{minipage}[h]{0.45\linewidth}
			\begin{itemize}
				\item<2 ->Information is included in a merkle tree
				\item<2 ->Merkle root is saved on-chain by a trusted party (e.g. government)
			\end{itemize}
		\end{minipage}
	\end{figure}

\end{frame}

%%%

\begin{frame}{Merkle Proofs}
	\begin{figure}
		\centering
		\resizebox{9cm}{!}{%
		\begin{tikzpicture}[
			basicnode/.style={rectangle, draw=black!60, fill=white!5, very thick, minimum size=5mm},
			bluenode/.style={rectangle, draw=black!60, fill=blue!7, very thick, minimum size=5mm},
			greennode/.style={rectangle, draw=black!60, fill=green!6, very thick, minimum size=5mm},
			basicdotted/.style={rectangle, draw=blue!60, fill=black!5, very thick, dashed, minimum size=5mm},
			]
			%Nodes
			\node[bluenode, align=center] (startpoint) 	at (6,1) {D:\\ 787793\\837929};
			\node[basicnode, align=center] (PlaceOfBirth)	[right=0.75cm of startpoint] {E:\\ Place of Birth:\\ Basel};
			\node[basicnode, align=center] (F)	[right= 0.25cm of PlaceOfBirth] {F:\\ 989229\\ 104925};
			\node[greennode, align=center] (DateOfBirth)	[left= 0.25cm of startpoint] {C:\\ Date of Birth:\\ 1/1/2000};
			\node[basicnode, align=center] (B)	[left= 0.75cmof DateOfBirth] {B:\\ 103982\\ 023492};
			\node[basicnode, align=center] (Name)	[left= 0.25cm of B] {A:\\ Name:\\ Alice};
			
			\node[bluenode, align=center] (HAB) at (0.2,3) {$H_{AB}$};
			\node[basicdotted, align=center] (HCD) at (4.6,3) {$H_{CD}$};
			\node[bluenode, align=center] (HEF)	at (9.7,3) {$H_{EF}$};
			
			\node[basicdotted, align=center] (HABCD) at (2.5,5) {$H_{ABCD}$};
			\node[basicdotted, align=center] (HABCDEF)	at (5.5,7) {Root\\ $H_{ABCDEF}$};	
			
			%Lines
			\draw[-stealth,thick,shorten >=0.05cm,shorten <=0.05cm] (Name.north) -- (HAB.south);
			\draw[-stealth,thick, shorten >=0.05cm,shorten <=0.05cm] (B.north) -- (HAB.south);
			\draw[-stealth,thick, shorten >=0.05cm,shorten <=0.05cm] (DateOfBirth.north) -- (HCD.south);
			\draw[-stealth,thick, shorten >=0.05cm,shorten <=0.05cm] (startpoint.north) -- (HCD.south);
			\draw[-stealth,thick, shorten >=0.05cm,shorten <=0.05cm] (PlaceOfBirth.north) -- (HEF.south);
			\draw[-stealth,thick, shorten >=0.05cm,shorten <=0.05cm] (F.north) -- (HEF.south);
			
			\draw[-stealth,thick, shorten >=0.05cm,shorten <=0.05cm] (HAB.north) -- (HABCD.south);
			\draw[-stealth,thick, shorten >=0.05cm,shorten <=0.05cm] (HCD.north) -- (HABCD.south);
			\draw[-stealth,thick, shorten >=0.05cm,shorten <=0.05cm] (HABCD.north) -- (HABCDEF.south);
			\draw[-stealth,thick, shorten >=0.05cm,shorten <=0.05cm] (HEF.north) -- (HABCDEF.south);
		\end{tikzpicture}}
	\end{figure}
	\begin{itemize}
		\item<2 -> Alice provides C, D, $H_{AB}$ and $H_{EF}$
		\item<3 -> Authenticity can be verified with Merkle root ($H_{ABCDEF}$) 
	\end{itemize}
\end{frame}

%%%

\begin{frame}{Tokenization}
	\center
	\begin{block}{\textbf{Definition}}
		Non-native tokens are rivalrous digital units which entitle its current owner to (the delivery of) an asset or service.
	\end{block}
	\vspace{1cm}
	\uncover<2 ->{\begin{figure}
		\begin{minipage}[t]{0.45\linewidth}
			\center
			\includegraphics[width=2cm]{../assets/images/house_1.png}
		\end{minipage}
		\hfill
		\begin{tikzpicture}[remember picture, overlay]
			\node[above] at (-0.3, 1.6){\scriptsize{Counterparty Risk}};
			\fill[highlight] (-1.3, 0.6) -- (0, 0.6) -- (0, 0.1) -- (0.5, 0.85) -- (0, 1.6) -- (0, 1.2) -- (-1.3, 1.2) -- (-1.3, 0.6);
		\end{tikzpicture}
		\begin{minipage}[t]{0.45\linewidth}
			\center
			\includegraphics[width=2.3cm]{../assets/images/token_house.png}
		\end{minipage}
	\end{figure}}
	\vspace{0.5cm}
	\begin{itemize}
		\item<3 ->Basically, every asset or promise can be tokenized
		\item<4 ->Differentiation between fungible and non-fungible tokens
	\end{itemize}
\end{frame}

%%%

\begin{frame}{Tokenization}
	\begin{columns}
		\begin{column}{0.6\textwidth}
			\textbf{Colored Coins}
			\begin{itemize}
				\item<2 -> External promise is attached to a Bitcoin UTXO
				\item<3 -> Additional output with meta data
				\item<4 -> Analogy: 
				\begin{itemize}
					\item<4 -> 5$\,$\$ bill
					\item<4 -> Promise is written on bill
					\item<4 -> Can freely circulate, including the promise
				\end{itemize}
			\end{itemize}
		\end{column}
		\begin{column}{0.4\textwidth}
			\vspace{-0.5cm}
			\begin{figure}
				\centering
				\begin{tikzpicture}[coloredCoin/.style={circle,  very thick, minimum size=5mm}]
					\node[coloredCoin, draw=green!60, fill=green!5] (green) at (1,1) {$\btc$};
					\node[coloredCoin, draw=magenta!60, fill=magenta!5] (magenta) at (3,0.9) {$\btc$};
					\node[coloredCoin, draw=teal!60, fill=teal!5] (teal) [above=0.7cm of green] {$\btc$};
					\node[coloredCoin, draw=darkgray!60, fill=darkgray!5] (darkgrey) at (0.3,1.7) {$\btc$};
					\node[coloredCoin, draw=orange!60, fill=orange!5] (orange) [below=1.2cm of darkgrey] {$\btc$};
					\node[coloredCoin, draw=olive!60, fill=olive!5] (olive) at (1.5,1.5) {$\btc$};
					\node[coloredCoin, draw=cyan!60, fill=cyan!5] (cyan) at (1.5,0.3) {$\btc$};
					\node[coloredCoin, draw=purple!60, fill=purple!5] [below=0.1cm of cyan] {$\btc$};
					\node[coloredCoin, draw=violet!60, fill=violet!5] at (2.3,2.8) {$\btc$};
				\end{tikzpicture}
			\end{figure}
		\end{column}
	\end{columns}
\end{frame}

%%%

\begin{frame}{Tokenization}
	\begin{columns}
		\begin{column}{0.65\textwidth}
			\textbf{Layer-based Tokens}
			\begin{itemize}
				\item<1 -> \texttt{OP\_RETURN} is used to save arbitrary data on-chain 
				\item<2 -> Transaction graph on a second layer is needed for interpretation
				\item<3 -> Analogy:
				\begin{itemize}
					\item<4 -> Transaction system based on newspapers
					\item<4 -> Promise (Token) is published in a encoded advertisement
					\item<4 -> External transaction graph interprets and registers transfer
				\end{itemize}
			\end{itemize}
		\end{column}
		\begin{column}{0.4\textwidth}
			\begin{figure}
				\centering
				\includegraphics[width = 3.5cm]{../assets/images/newspaper.jpg}
			\end{figure}
		\end{column}
	\end{columns}
\end{frame}

%%%

\begin{frame}{Tokenization}	
	\begin{itemize}
		\item<1 ->\textbf{Colored Coins}
		\begin{itemize}
			\item<1 ->Use a fragment of a native Blockchain asset (UTXO) as a container
		\end{itemize}
		\vspace{1em}
		\item<2 ->\textbf{Layer-based Tokens}
		\begin{itemize}
			\item<2 -> Use metadata transactions (\texttt{OP\_RETURN}) and a separate transaction graph to create and track tokens
		\end{itemize}
		\vspace{1em}
		\item<3 ->\textbf{Smart Contract-based Token}
		\begin{itemize}
			\item<3 -> A dedicated smart contract creates and tracks states that represent token ownership. It maps tokens to current owner addresses.
		\end{itemize}
	\end{itemize}
\end{frame}

%%%

\begin{frame}{Smart Contracts}
	\begin{block}{\textbf{\textcolor{black}{Definition}}}
		A smart contract is a semi-automated process which is secured by a distributed consensus protocol.
	\end{block}
	\begin{itemize}
		\item<2-> Definition not tied to a specific Blockchain
		\item<3-> Not smart, not a contract, not automated
		\item<4-> a
	\end{itemize}
\end{frame}

%%%

\begin{frame}[fragile]{Smart Contracts}
	\begin{columns}
		\begin{column}{0.3\textwidth}
			\includegraphics[width=1\textwidth]{../assets/images/vending-machine.png}
		\end{column}
		\begin{column}{0.5\textwidth}
			\begin{block}{\textcolor{black}{\textbf{Simple Vending Machine}}}
				\begin{lstlisting}[firstnumber=1,  xleftmargin=0pt, columns=fullflexible,language=Solidity] 
if(coin >= price){
	dispenseBeverage();
	returnChange(coin - price);
}else{
	print("insufficient funds");
}
				\end{lstlisting}
			\end{block}
		\end{column}
	\end{columns}
	\vspace{0.5cm}
	\uncover<2 ->{
		Smart Contracts:
		\begin{itemize}
			\item Plain code is not open source
			\item Execution environment is not visible
	\end{itemize}}
\end{frame}

%%%

\begin{frame}{Smart Contracts}
	UTXO vs Account based
\end{frame}

%%%

%\begin{frame}%[allowframebreaks]
%	\frametitle{References}
%	\bibliographystyle{amsplain}
%	\bibliography{../assets/bib/refs}
%\end{frame}

%%%
\end{document}