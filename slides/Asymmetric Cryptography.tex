% Choose one to switch betweeen slides and handout
\documentclass[]{beamer}
%\documentclass[handout]{beamer}

% Video Meta Data
\title{Bitcoin, Blockchain and Cryptoassets}
\subtitle{Intro: Welcome to the Course}
\author{Prof. Dr. Fabian Schär}
\institute{University of Basel}

% Config File
% Packages
\usepackage[utf8]{inputenc} 
\usepackage{hyperref}
\usepackage{gitinfo2}
\usepackage{tikz}
\usepackage{amsmath}
\usepackage{bibentry}
\usepackage{xcolor}
\usepackage{caption}

% Beamer Template Options
\beamertemplatenavigationsymbolsempty
\setbeamertemplate{footline}[frame number]
\setbeamercolor{structure}{fg=black}
\setbeamercolor{footline}{fg=black}
\setbeamercolor{title}{fg=black}
\setbeamercolor{frametitle}{fg=black}
\setbeamercolor{item}{fg=black}
\setbeamercolor{}{fg=black}
\setbeamercolor{bibliography item}{fg=black}
\setbeamercolor*{bibliography entry title}{fg=black}
\setbeamertemplate{items}[square]
\setbeamertemplate{enumerate items}[default]
\captionsetup[figure]{labelfont={color=black},font={color=black}}
\captionsetup[table]{labelfont={color=black},font={color=black}}

\setbeamertemplate{bibliography item}{\insertbiblabel}

% Link Icon Command
\newcommand{\link}{%
    \tikz[x=1.2ex, y=1.2ex, baseline=-0.05ex]{% 
        \begin{scope}[x=1ex, y=1ex]
            \clip (-0.1,-0.1) 
                --++ (-0, 1.2) 
                --++ (0.6, 0) 
                --++ (0, -0.6) 
                --++ (0.6, 0) 
                --++ (0, -1);
            \path[draw, 
                line width = 0.5, 
                rounded corners=0.5] 
                (0,0) rectangle (1,1);
        \end{scope}
        \path[draw, line width = 0.5] (0.5, 0.5) 
            -- (1, 1);
        \path[draw, line width = 0.5] (0.6, 1) 
            -- (1, 1) -- (1, 0.6);
        }
    }

% Custom Titlepage
\defbeamertemplate*{title page}{customized}[1][]
{
  \vspace{-0cm}\hfill\includegraphics[width=2.5cm]{../config/logo_cif} 
  \includegraphics[width=1.9cm]{../config/seal_wwz} 
  \\ \vspace{2em}
  \usebeamerfont{title}\textbf{\inserttitle}\par
  \usebeamerfont{title}\usebeamercolor[fg]{title}\insertsubtitle\par  \vspace{1.5em}
  \small\usebeamerfont{author}\insertauthor\par
  \usebeamerfont{author}\insertinstitute\par \vspace{2em}
  \usebeamercolor[fg]{titlegraphic}\inserttitlegraphic
    \tiny \noindent \texttt{Commit Hash: \gitHash}\\ 
	\texttt{Commit Time: \gitAuthorIsoDate}\\ \vspace{1em}
  \link \href{https://github.com/cifunibas/Bitcoin-Blockchain-Cryptoassets/blob/main/slides/intro.pdf}
  {Get most recent version}\\
  \link \href{https://github.com/cifunibas/Bitcoin-Blockchain-Cryptoassets/blob/main/slides/intro.pdf}
  {Watch video lecture}\\ \vspace{1em}
  License: \texttt{Creative Commons Attribution-NonCommercial-ShareAlike 4.0 International}\\\vspace{2em}
  \includegraphics[width = 1.2cm]{../config/license}
}


%%%%%%%%%%%%%%%%%%%%%%%%%%%%%%%%%%%%%%%%%%%%%%
%%%%%%%%%%%%%%%%%%%%%%%%%%%%%%%%%%%%%%%%%%%%%%
\begin{document}

\thispagestyle{empty}
\begin{frame}[noframenumbering]
	\titlepage
\end{frame}

%%%
\begin{frame}{Diffie-Hellman-Merkle Key Exchange}
	\begin{itemize}
		\item<1-> Whitfield Diffie, Martin Hellman and Ralph Merkle find a solution for the \textit{key distribution} problem (1976)
		\item<2-> They still use symmetric encryption, but the common key can be communicated on a public channel
		\vspace{0.5cm}
		\item<3-> Alice and Bob define ex-ante
		\begin{itemize}
			\item<4-> One-way function: \(Y^x\ (mod\ P)\)
			\item<5-> Parameters: \(Y = 7\) and \(P = 11\)
		\end{itemize}
	\end{itemize}
\end{frame}
%%%

%%%
\begin{frame}{Diffie-Hellman-Merkle Key Exchange}
	\begin{table}
		\centering
		\resizebox{10.5cm}{!}{
			\begin{tabular}{c c c c}
				\hline
				&Alice&&Bob\\
				\hline
				&&&\\
				Step 1 & chooses \(A = 3\) && chooses \(B = 6\)\\
				&&&\\
				Step 2 & \(\alpha = Y^A\ (mod\ P)\) && \(\beta = Y^B\ (mod\ P)\) \\					& \(\alpha = 7^3\ (mod\ 11) = 2\) &&\(\beta = 7^6\ (mod\ 11) = 4\) \\
				&&&\\
				Step 3 & Alice sends her result \(\alpha = 2\) to Bob && Bob sends his result \(\beta = 4\) to Alice \\
				&&&\\
				\textit{Exchange} & \multicolumn{3}{c}{\textit{The only exchange of information (not key!)}}\\
				&&&\\
				Step 4 & Alice calculates \(k = \beta^A\ (mod\ P)\) && Bob calculates \(k = \alpha^B\ (mod\ P)\) \\
				&\(k = 4^3\ (mod\ 11) = 64\ (mod\ 11) = 9\) && \(k = 2^6\ (mod\ 11) = 64\ (mod\ 11) = 9\) \\
				&&&\\
				\textit{Key} & \multicolumn{3}{c}{\textit{Alice and Bob have agreed on key \(k = 9\), because:}}\\
				& \multicolumn{3}{c}{$\alpha^{B}\ (mod\ P) = Y^{AB}\ (mod\ P) = Y^{BA}\ (mod\ P) = \beta^{A}\ (mod\ P)$}\\
				&&&\\
				\hline
			\end{tabular}
			}
		\caption{Diffie-Hellman-Merkle-Key-Exchange. Based on \cite{singh1999}}
	\end{table}
\end{frame}
%%%

%%%
\begin{frame}{Issues}
	\vspace{-0.3cm}
	\begin{figure}
		\centering
		\includegraphics[width = 9cm]{../assets/images/drawing_DHM_exchange.png}
	\end{figure}
	\vspace{0.5cm}
	\begin{itemize}
		\item<1-> Undermines spontaneity of online communication
	\end{itemize}
\end{frame}
%%%

%%%
\begin{frame}{Asymmetric Cryptography}
	\begin{itemize}
		\item<1-> So far: Symmetric cryptography
		\begin{itemize}
			\item<1->Decryption is inverse algorithm of encryption
		\end{itemize}
		\item<2-> Diffie's Idea (1975): Asymmetric encryption
		\begin{itemize}
			\item<2->Key to encrypt and key to decrypt a message are not identical
		\end{itemize}
		\item<3->...but how?
	\end{itemize}
\end{frame}
%%%

%%%
\begin{frame}{Asymmetric Cryptography}
	\begin{itemize}
		\item<1-> Ronald Rivest, Adi Shamir and Leonard Adleman
		\item<2-> Inspired by Diffie's idea of asymmetric cryptography
		\item<3-> Develop first asymmetric encryption algorithm (\textit{RSA}) \cite{rivest1978}
	\end{itemize}
	\uncover<3->{
		\begin{figure}
			\centering
			\includegraphics[width = 8cm]{../assets/images/cover_RSA}
		\end{figure}
	}
\end{frame}
%%%

%%%
\begin{frame}{RSA}
	Properties of RSA
	\begin{itemize}
		\item<1->[(a)] Alice creates a \textit{public key} with which everyone can encrypt a message. The encryption with the public key is a one-way function and therefore not revertable
		\item<2->[(b)] Only Alice is able to decrypt the message with her \textit{private key} and revert the encryption process
	\end{itemize}
\end{frame}
%%%

%%%
\begin{frame}{RSA 1/4: Find Public Key}
	Choose two primes \(p\) and \(q\) (e.g. \(p = 17\) and \(q = 11\)) and calculate
	\begin{align*}
		N &= p \cdot q \\
		\intertext{In our example:}
		\begin{split}
			N &= 17 \cdot 11 = 187 
		\end{split}
	\end{align*} 
	and choose an additional integer \(e\) (e.g. \(e = 7\)).\footnote{\tiny \(e\) and \((p-1)\cdot(q-1)\) have to be relative primes} \\
	\vspace{0.5cm}
	\(\Rightarrow\) \(N\) and \(e\) together make up the Public Key
\end{frame}
%%%

%%%
\begin{frame}{RSA 2/4: Encrypt Message}
	Convert message \(M\) into an integer (e.g. \(M =\) letter \(X\) in ASCII \(= 88\)) and calculate the encrypted message \(C\) using the formula 
	\begin{align*}
		C &= M^e\ (mod\ N) \\
		\intertext{In our example:}
		C &= 88^7\ (mod\ 187) \\
		C &= 11
	\end{align*}
\end{frame}
%%%

%%%
\begin{frame}{RSA 3/4: Derive Private Key}
	Calculate Private Key \(k_{p}\) using the formula
	\begin{align*}
		e \cdot k_{p} &= 1\ (mod\ \phi(N))\;\text{, where \(\phi(N) = (p-1)(q-1)\)}
		\intertext{In our example:}
		7 \cdot k_{p} &= 1\ (mod\ 16 \cdot 10) = 1\ (mod\ 160) \\
		k_{p} &= 23
	\end{align*}
	\uncover<2->{Note: Extended Euclidean algorithm is needed for this step}
\end{frame}
%%%

%%%
\begin{frame}{RSA 4/4: Decrypt Message}
	\begin{align*}
		\intertext{Decrypt with formula}
		M &= C^{k_{p}}\ (mod\ N)
		\intertext{Insert and calculate}
		M &= 11^{23}\ (mod\ 187) \\
		M &= 88 = X
		\text{\ in ASCII}
	\end{align*}
\end{frame}
%%%

%%%
\begin{frame}{RSA: Requirements}
	Crucial: 
	\begin{itemize}
		\item<1-> \(N\) has to be sufficiently large
		\item<2-> It has to be impossible to find \(p\) and \(q\) by using prime factorization of \(N\).
	\end{itemize}
\end{frame}
%%%

%%%
\begin{frame}{Pretty Good Privacy}
		\begin{itemize}
			\item<1->RSA is computationally costly
			\item<2->General public did not benefit 
			\item<3->Phil Zimmermann wanted to protect privacy of the average citizen on the internet as well
			\item<4->His solution: Pretty Good Privacy (PGP)
		\end{itemize}
\end{frame}
%%%

%%%
\begin{frame}{PGP: Principle}
	\begin{minipage}[T][5cm][t]{0.45\textwidth}
		Encryption:
		\begin{itemize}
			\item<1->Encryption of message is done symmetrically
			\item<2->Encryption of symmetric key using RSA-public-key of recipient
			\item<3->Sending encrypted message and symmetric key to recipient
		\end{itemize}
	\end{minipage}
	\hfill
	\begin{minipage}[T][5cm][t]{0.45\textwidth}
		\uncover<4->{Decryption:}
		\begin{itemize}
			\item<4->Decryption of symmetric key with RSA-private-key of recipient.
			\item<5->Decryption of message with decrypted symmetric key
		\end{itemize}
	\end{minipage}
\end{frame}
%%%

%%%
\begin{frame}%[allowframebreaks]
	\frametitle{References}
	\bibliographystyle{amsplain}
	\bibliography{../assets/bib/refs}
\end{frame}
%%%
\end{document}