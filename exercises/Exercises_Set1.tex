% !TEX TS-program = pdflatex
% !TEX encoding = UTF-8 Unicode


\documentclass[12pt]{article} 

\usepackage[utf8]{inputenc} % set input encoding 
\usepackage{geometry} % to change the page dimensions
\geometry{a4paper} 
\usepackage[parfill]{parskip} % No indent in document

% Packages
\usepackage{graphicx}
\usepackage{array}
\usepackage{amsmath}
\usepackage{verbatim}
\usepackage{xcolor}
\usepackage[hidelinks]{hyperref}
\usepackage{tikz}

\usepackage{todonotes}
\setuptodonotes{inline, size=\small}


\title{Bitcoin, Blockchain and Cryptoassets \\ Exercise Set 1}
\author{}
\date{ } 

\begin{document}
	\maketitle
	
	In this exercise set you will get a chance to train how to convert between different numeral systems, try to break a symmetric cipher and use the RSA algorithm to encrypt and decrypt a message.
	\par
	
	Keep in mind that solving this exercise set is voluntary and UNGRADED. The solutions are either shared already or will be in due time.
	

	\newpage
	
	\section*{Exercise 1}
	\label{sec:numeralSystems}
	This first exercise aims at making you familiar with the conversion between different numeral systems.
	\subsection*{Exercise 1.1}
	
	\begin{itemize}
		\item[a)] Convert the binary number $({\tt10110101})_{B=2}$ into a decimal number.
		\item[b)] Conver the decimal number $(93)_{B = 10}$ into a binary number.
	\end{itemize}

	\subsection*{Exercise 1.2}
	
	\begin{itemize}
		\item[a)] Convert the binary number $({\tt10110101})_{B=2}$ into a hexadecimal (with basis $B = 16$). 
		\item[b)] Convert the hexadecimal $($9c3a$)_{B = 16}$ into a binary number. 
	\end{itemize}

	\section*{Exercise 2}
	\label{sec:monoalphabetic}
	The following (English) text was encrypted with the monoalphabetic substitution cipher:
	\par
	RQL IFMXAXQKNY XQKFFLYZL NH K ZJLKR MOOMJRGYNRD RM ZKNY OJKXRNXKF LPOLJNLYXL TGJNYZ DMGJ HRGTNLH KYT RM UMJA MY K FKJZLJ OJMVLXR RMZLRQLJ UNRQ K XMEOKYD.\\
	RQL HRGTLYR RLKEH TLHNZY K HMFGRNMY MY RQL IKHNH MC K OJKXRNXL-MJNLYRLT OJMIFLE TLHXJNORNMY. KXXMEOKYNLT ID XMKXQLH KYT LPOLJRH, RQLD ZM RQJMGZQ RQL HRLOH NY XJLKRNYZ K XMYXLOR NYXFGTNYZ K OJMRMRDOL. RQNH KNEH KR HRJLYZRQLYNYZ RQLNJ RLKEUMJA, MJZKYNHKRNMYKF KYT OJLHLYRKRNMY HANFFH, KH ULFF KH NEOJMWNYZ RNEL EKYKZLELYR KYT OJMIFLE-HMFWNYZ XMEOLRLYXL. CGJRQLJEMJL, RQL XQKFFLYZL MCCLJH K GYNBGL XQKYXL RM YLRUMJA UNRQ XMEOKYNLH KYT LPOLJRH KH ULFF KH HRGTLYRH UQM KJL NYRLJLHRLT NY RQL HKEL RMONXH.\par
	
	% The Blockchain Challenge is a great opportunity to gain practical experience during your studies and to work on a larger project together with a company.
	
	% The student teams design a solution on the basis of a practice-oriented problem description. Accompanied by coaches and experts, they go through the steps in creating a concept including a prototype. This aims at strengthening their teamwork, organisational and presentation skills, as well as improving time management and problem-solving competence. Furthermore, the Challenge offers a unique chance to network with companies and experts as well as students who are interested in the same topics.
	
	Decrypt it using frequency analysis and by exploiting the relations between letters. (Hint: Use frequency analysis only to decipher the three most frequent letters. Also, there are online tools which can help you with finding the relative frequencies of the letters\footnote{E.g. \url{https://www.mtholyoke.edu/courses/quenell/s2003/ma139/js/count.html}}).
	
	\newpage
	
	\section*{Exercise 3}
	\label{sec:RSA}
	In this exercise we use the RSA algorithm covered in the lecture on asymmetric cryptography to create a private and public key and send an encrypted message along the lines of the example in the lecture. We use the parameters $p = 11$, $q = 23$ and $e = 7$ to encipher the message $L$.
	
	\subsection*{Exercise 3.1}
	
	Compute the number $N$ and convert the message $M = L$ using the ASCII table into a decimal number (!). Thereafter encrypt this decimal number to receive the encrypted message $C$.\footnote{Conventional calculators might not be able to handle numbers of this size well. We recommend to use the online modulo calculator of \href{https://planetcalc.com/8326/}{https://planetcalc.com/8326/}.}
	
	\subsection*{Exercise 3.2}
	
	Compute the number $\phi(N)$ and derive the private key $k_{p}$, which will be used to decrypt the message $C$. 
	\\
	\\
	Hint: To find the multiplicative inverse in modulo-calculations we have to use the Euklidean Algorithm. There are several online tools which you can use, instead of doing this by hand\footnote{A simple tool would be the inverse calculator by \href{https://planetcalc.com/3311/}{https://planetcalc.com/3311/}. To derive the private key ($k_p$) in the RSA example in the lecture on asymmetric cryptography you would have to insert 7 und "Integer" and 160 under "Modulo". The result is the private key $k_{p}$. Use the analogous procedure in this exercise.}.
	
	\subsection*{Exercise 3.3}
	
	Decrypt the encrypted message $C$ using the private key $k_{p}$ and check the result with the original message $M = L$.	
	

	\newpage

	\section*{Exercise 4}

The following (\textcolor{blue}{blue text}) is encrypted with the same cipher alphabet as used in Exercise 2. Punctuation marks are not encrypted. Furthermore you will need the RSA algorithm with the parameters from Exercise 3, the ASCII conversion table from the lecture on symmetric cryptography, and you must be able to convert between different numeral systems, as covered in Exercise 1. Decrypting this text will give you all the informations you need to come up with the solution.\\
	
	\textcolor{blue}{RQL ALD UMJT NH RQL} 01001100010000010101001101010100 \textcolor{blue}{UMJT MY OKZL} 192 \textcolor{blue}{NY RQL LYZFNHQ INRXMNY UQNRLOKOLJ MY INRXMNY.MJZ OFGH RQL TLXNEKF JLOJLHLYRKRNMY MC RQL QLPKTLXNEKF }4b.\par
	%The key word is the last word on page 4 in the English bitcoin whitepaper on bitcoin.org.
	% plus the decimal representation of the hexadecimal 4b (75).
	
	\vspace{0.5cm}
	To verify if you have found the correct solution, concatenate (no spaces) the word and the number, compute its SHA256 hash value using an online hash calculator and compare it to the hexadecimal string below. Example: If your solution is "bitcoin" and "21" you would have to compute the SHA256 of ``bitcoin21''. Make sure you write the word in lowercase letters and add no spaces. As you have learned in the lecture, even the smallest change in the input will lead to an entirely different hash value. The SHA256 hash value of the correct solution is:
	\begin{center}
		%c064fbca9d9de8dd9bb0624984403b28d0da807a69365d4f7fb09123ecb0c405 % memory
		75324da9af9da0e9ea77abc4a0c46afffd7e6f80080652e149fc19f26a10b97f % memory75
	\end{center}
	
\end{document}